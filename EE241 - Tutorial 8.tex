\documentclass{tutorial}
\begin{document}
\newif\ifsolns

%%%%%%%%%%%%%%%%%%%%%%%%%
% UNCOMMENT BELOW TO TURN ON SOLNS %
%%%%%%%%%%%%%%%%%%%%%%%%%
\solnstrue

\title{EE241 Spring 2015: Tutorial \#8}
\date{Friday, March 13, 2015}
\maketitle

\begin{prob}
If $\vec{u}$ and $\vec{v}$ are related by angle $\theta$, what is the angle between $\vec{u} / | \vec{u} |$ and $\vec{v} / |\vec{v}|$? What is the angle between $\vec{u} / |\vec{u}|$ and $\vec{v}$?
\end{prob} \ifsolns \begin{proof}
The angle between $\vec{u} / | \vec{u} |$ and $\vec{v} / |\vec{v}|$ is given by
\[
  \frac{\displaystyle\frac{\vec{u}}{|\vec{u}|} \cdot \displaystyle\frac{\vec{v}}{|\vec{v}|}}
  {\left| \displaystyle\frac{\vec{u}}{|\vec{u}|} \right| \left|\displaystyle\frac{\vec{v}}{|\vec{v}|} \right|}
  = 
  \frac{\vec{u}}{|\vec{u}|} \cdot \frac{\vec{v}}{|\vec{v}|}
  \lp \frac{|\vec{u}|}{|\vec{u}|} \frac{|\vec{v}|}{|\vec{v}|} \rp^{-1}
  =
  \frac{\vec{u}}{|\vec{u}|} \cdot \frac{\vec{v}}{|\vec{v}|}
  = \cos \theta .
\]
The angle between $\vec{u} / |\vec{u}|$ and $\vec{v}$ is given by
\[
  \frac{\displaystyle\frac{\vec{u}}{|\vec{u}|} \cdot \vec{v}}
  {\left| \displaystyle\frac{\vec{u}}{|\vec{u}|} \right| \left| \vec{v} \right|}
  = 
  \frac{\vec{u}}{|\vec{u}|} \cdot \frac{\vec{v}}{|\vec{v}|}
  \lp \frac{|\vec{u}|}{|\vec{u}|} \rp^{-1}
  =
  \frac{\vec{u}}{|\vec{u}|} \cdot \frac{\vec{v}}{|\vec{v}|}
  = \cos \theta .
\]
Thus, normalizing either vector does not change the angles between them.
\end{proof}\else \vspace{3in} \fi



\begin{prob}
Find the set of all $3$-dimensional vectors that are $30^{\circ}$ away from $\vec{u} = \ls 1, 0, 1 \rs^T$ and orthogonal to $\vec{v} = \ls -3, 1, 0 \rs^T$. \textbf{Hint:} Start by finding the set of all length-$1$ vectors (unit vectors) that satisfy these conditions.
\end{prob} \ifsolns \begin{proof}
Let $\vec{s} = \ls a, b, c \rs^T$ be any length-$1$ vector. Now apply the first condition
\[
  \frac{\vec{u} \cdot \overrightarrow{s}}{| \vec{u} | | \overrightarrow{s} |} = \frac{a+c}{\sqrt{2}}
\]
Note that $\cos \lp 30^{\circ} \rp = \sqrt{3}/2$, thus
\[
  a+c = \sqrt{3/2}.
\]
The second condition is that
\[
  \vec{v} \cdot \overrightarrow{s} = 0
  \hspace{0.25in} \Longrightarrow \hspace{0.25in}
  -3a + b = 0 .
\]
Altogether
\[
  \ls \begin{array}{rrr}
     1 &  0 &  1 \\
    -3 &  1 &  0
  \end{array} \rs
  \ls \begin{array}{c} a \\ b \\ c \end{array} \rs
   = \ls \begin{array}{r} \sqrt{3/2} \\ 0 \end{array} \rs
   \hspace{0.25in} \Longrightarrow \hspace{0.25in}
   \ls \begin{array}{rrr}
    1 &  0 &  1 \\
    0 &  1 &  3
  \end{array} \rs
  \ls \begin{array}{c} a \\ b \\ c \end{array} \rs
   = \ls \begin{array}{r} \sqrt{3/2} \\ 3\sqrt{3/2} \end{array} \rs.
\]
The solutions to this system are parametrized by $t$
\[
  \ls \begin{array}{c} a \\ b \\ c \end{array} \rs
  =
  \ls \begin{array}{c} \sqrt{3/2} \\ 3\sqrt{3/2} \\ 0 \end{array} \rs
  + t \ls \begin{array}{c} -1 \\ -3 \\ 1 \end{array} \rs .
\]
Of course, the vector above is not length-$1$ for all values of $t$. However, we can now drop this requirement using arguments from Problem 1. The set of vectors that satisfy the conditions is just
\[
  S = \lb \ls \sqrt{3/2} , 3 \sqrt{3/2}, 0 \rs + t \ls -1, -3, 1 \rs \; : \; t \in \mbR \rb
\]
\end{proof}\else \newpage \fi



\begin{prob} Find the approximate angle between the following two $100$-dimensional vectors
\begin{align*}
  \vec{u} & = \ls 1, -1,  1, -1,  1, -1, \dots, 1, -1,  1, -1 \rs \\
  \vec{v} & = \ls 1,1/2,1/4,1/8,1/16,1/32\dots, 2^{-96}, 2^{-97},  2^{-98}, 2^{-99}\rs.
\end{align*}
What happens when the dimension of the above vectors grows (respecting the patterns above)?
\end{prob} \ifsolns \begin{proof}
First, note that
\begin{align*}
  |\vec{u}| & = \sqrt{100} \\
            & = 10 \\
  |\vec{v}| & = \sqrt{\sum_{n=0}^{99} \lp\frac{1}{2^2}\rp^n} \\
            & = \sqrt{\frac{1-(1/4)^{1000}}{1-1/4}} \\
            & \approx 2/\sqrt{3}
\end{align*}
Their inner product yields
\begin{align*}
  \vec{u} \cdot \vec{v}
  & = \sum_{n=0}^{99} \lp -1/2 \rp^n \\
  & = \frac{1 - (-1/2)^{100}}{1 + 1/2} \\
  & \approx 2/3
\end{align*}
Altogether
\[
  \cos \theta \approx \frac{2/3}{10 \cdot 2/\sqrt{3}} = \frac{1}{\sqrt{3} \cdot 10}
  \hspace{0.15in} \Longrightarrow \hspace{0.15in}
  \theta \approx 86.7^{\circ}
\]
As the dimension of $\vec{u}$ and $\vec{v}$ grows, their internal angle approaches $90^{\circ}$ and the vectors becomes nearly orthogonal.
\end{proof}\else \vspace{3in} \fi



\begin{prob} Consider the following linear transformation
\[
  L \lp \ls \begin{array}{c} w \\ x \\ y \\ z \end{array} \rs\rp
  = \ls \begin{array}{c}
    w+z \\
    x+y \\
    w+x+y+z \\
    w-x-y+z
  \end{array} \rs
\]
What is the dimension of the range of $L$? How many equations define such a space? Give these equations.
\end{prob} \ifsolns \begin{proof}
The dimension of the range of $L$ is equal to the number of pivots that the matrix associated with $L$ contains. The matrix in question is
\[
  A_L = \ls \begin{array}{rrrr}
     1 &  0 &  0 &  1 \\
     0 &  1 &  1 &  0 \\
     1 &  1 &  1 &  1 \\
     1 & -1 & -1 &  1
  \end{array} \rs
\]
By doing some quick row manipulations we can see that the reduced row echelon form of this matrix is
\[
  A_L = \ls \begin{array}{rrrr}
     1 &  0 &  0 &  1 \\
     0 &  1 &  1 &  0 \\
     0 &  0 &  0 &  0 \\
     0 &  0 &  0 &  0
  \end{array} \rs .
\]
Thus the range of $L$ is a $2$-dimensional space. Since we are in a $4$-dimensional space, we will need two equations to restrict $\mbR^4$ to the range of $L$. In order to find a description of this space, consider using the Gauss-Jordan method on an augmented matrix
\begin{align*}
  \ls A_L \; | \; I \rs
  & = \ls \begin{array}{rrrr|rrrr}
     1 &  0 &  0 &  1 & 1 & 0 & 0 & 0 \\
     0 &  1 &  1 &  0 & 0 & 1 & 0 & 0 \\
     1 &  1 &  1 &  1 & 0 & 0 & 1 & 0 \\
     1 & -1 & -1 &  1 & 0 & 0 & 0 & 1 \\
  \end{array} \rs \\
  & \rightarrow \ls \begin{array}{rrrr|rrrr}
     1 &  0 &  0 &  1 & 1 & 0 & 0 & 0 \\
     0 &  1 &  1 &  0 & 0 & 1 & 0 & 0 \\
     0 &  0 &  0 &  0 &-1 &-1 & 1 & 0 \\
     0 &  0 &  0 &  0 &-1 & 1 & 0 & 1 \\
  \end{array} \rs
\end{align*}
Let's identify the left and right parts of the above matrix as $A_L'$ and $B_L$ (Note that $A_L = B_L^{-1}A_L'$). If we let $\ls a, b, c, d \rs$ be a vector in the range of $L$ then we have the relation
\[
  A_L' \ls \begin{array}{c} w \\ x \\ y \\ z \end{array} \rs
  = B_L \ls \begin{array}{c} a \\ b \\ c \\ d \end{array} \rs
\]
This reduces to
\[
  \ls \begin{array}{c}
    w+z \\
    x+y \\
    0 \\
    0
  \end{array} \rs
  = \ls \begin{array}{c}
    a \\
    b \\
    -a-b+c \\
    -a+b+d
  \end{array} \rs .
\]
Thus, we can simply read off the last two equations since they do not contain $w$, $x$, $y$, or $z$. The equations that define our space are
\begin{align*}
  0 & = -a-b+c \\
  0 & = -a+b+d
\end{align*}
\end{proof}\else \newpage \fi



\begin{prob} Find the plane passing through the following three points $P_1 = \ls 1,1,1 \rs$, $P_2 = \ls 1,2,3 \rs$, and $P_3 = \ls 2,1,1 \rs$. Write it in a parametrized form and then again in terms of constrains.
\end{prob} \ifsolns \begin{proof}
Consider answering the question, ``how do I get to a point in the plane from the origin?''. First, you must traverse from the origin to one of the points, say $P_1$, and then walk in the direction of $P_2$ or $P_3$. Thus, any point in the plane can be written as
\[
  \vec{x} = \overrightarrow{P_1} + s \overrightarrow{P_1P_2} + t \overrightarrow{P_1P_3}
\]
or
\[
  P = \lb \ls \begin{array}{c} 1 + t \\ 1+s \\ 1+2s \end{array} \rs \; : \; \forall s,t \in \mbR \rb .
\]
To write the constraint equation for this plane consider the following: take any point in the plane $Q$ and draw a vector to $P_1$. This vector must be perpendicular to the normal vector of the plane. The normal vector of the plane is found by taking the cross product of $\overrightarrow{P_1P_2}$ with $\overrightarrow{P_1P_3}$,
\begin{align*}
  \vec{n}
  & = \overrightarrow{P_1P_2} \times \overrightarrow{P_1P_3} \\
  & = \left| \begin{array}{rrr}
    \mathbf{i} & \mathbf{j} & \mathbf{k} \\
    0 & 1 & 2 \\
    1 & 0 & 0
  \end{array} \right| \\
  & = \ls 0, 2, -1 \rs^T
\end{align*}
Now, if $Q$ is at location $\ls x, y, z \rs^T$, then $\overrightarrow{P_1Q} = \ls x-1, y-1, z-1 \rs^T$. In turn, this means that the condition for our plane is $\overrightarrow{P_1Q} \cdot \vec{n} = 0$, or
\[
  2y - z = 1
\]
\end{proof}\else \vspace{4in} \fi



\begin{prob} A massive solar sail is deployed in space. The sail is shaped like a triangle and supported by three satellites. The satellites are reporting their positions relative to the international space station in metres. Satellite $1$ is at $\ls -1000, 1500, 2000\rs$, satellite $2$ is at $\ls 1000, 1500, 0 \rs$, satellite $3$ is at $\ls 500, -500, -500 \rs$. What is the area of the solar sail? \textbf{Hint:} Consider a parallelogram-shaped solar sail first.
\end{prob} \ifsolns \begin{proof}
First, let's work in kilometers for easier notation, this means the satellites are at
\begin{align*}
  S_1 & = \ls  -1, 3/2,   2 \rs \\
  S_2 & = \ls   1, 3/2,   0 \rs \\
  S_3 & = \ls 1/2,-1/2,-1/2 \rs
\end{align*}
If we consider a parallelogram-shaped sail, we can calculate it's area using the cross-product rule and the two vectors $\vec{u} = \overrightarrow{S_1S_2}$ and $\vec{v} = \overrightarrow{S_1S_3}$. First,
\begin{align*}
  \vec{u} & = \ls   2,  0,  -2 \rs \\
  \vec{v} & = \ls 3/2, -2,-5/2 \rs .
\end{align*}
We can use the determinant method for calculating the cross product,
\begin{align*}
  \left| \begin{array}{rrr}
    \mathbf{i} & \mathbf{j} & \mathbf{k} \\
    2 & 0 & -2 \\
    3/2 & -2 & -5/2
  \end{array} \right|
  & = \mathbf{i} \lp 0 \cdot (-5/2) - (-2) \cdot (-2) \rp \\
  & - \mathbf{j} \lp 2 \cdot (-5/2) - (3/2) \cdot (-2) \rp \\
  & + \mathbf{k} \lp 2 \cdot (-2) - (3/2) \cdot 0 \rp \\
  & = \ls -4, -2, -4 \rs .
\end{align*}
The area of a parallelogram sail is then $| \vec{u} \times \vec{v} | = 6 \text{km}^2$. Since a triangle sail would be half of this area then the area of the solar sail is $\boxed{3 \text{km}^2}$.
\end{proof}\else \newpage \fi









\end{document}